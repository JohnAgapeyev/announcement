%% LyX 2.3.0 created this file.  For more info, see http://www.lyx.org/.
%% Do not edit unless you really know what you are doing.
\documentclass[english]{article}
\usepackage[T1]{fontenc}
\usepackage[latin9]{inputenc}
\usepackage{float}
\usepackage{algorithm}
\usepackage{babel}

\begin{document}

\title{A new belief announcement solver}

\author{Aaron Hunter, John Agapeyev}
\maketitle
\begin{abstract}
We present a new tool for solving the problem of belief revision across
multiple agents using a shared announcement formula. We model the
problem as a deepest-node search inside a multitree, where the depth
represents the number of agents satisfied by a given formula. We hope
modeling the problem in this way will allow future improvements
\end{abstract}

\section{Introduction}

Belief revision refers to the process where an agent incorporates
new information to an existing set of beliefs. For the sake of efficiency,
it may be desirable to perform belief revision in bulk across multiple
agents to accomplish multiple goals. The key problem behind this idea
of shared bulk revision is finding a suitable announcement. Belief
Revision is already a known hard problem, so the goal is to minimize
overhead in solving the problem across multiple agents. We present
a new tool to solve this problem in an innovative and efficient way
that lays the groundwork for future improvements that have been applied
to similar problem modelling schemes.

\section{Background}

I don't understand the problem enough to write background on it.

\section{Design}

We model the problem of finding a suitable announcement for \emph{N}
agents as exploring a multitree, where each node is a possible announcement
formula, and its depth represents the number of agents whose goals
are satisfied by the formula. In this way, the problem can be modeled
as a deepest-node search inside the multitree, where we want to find
the node farthest from the roots, as it would represent either a solution
for all \emph{N }agents, or it would fail to reach the maxmium depth,
indicating the problem has no solution. The idea behind modeling the
problem in this fashion is to potentially benefit from algorithmic
improvements in future work such as heuristics and backtracking that
can be found in algorithms such as Conflict Driven Clause Learning
and Alpha-Beta pruning.

\begin{algorithm}[H]
\caption{Belief Announcement}

\textbf{function }FindAnnouncement($\beta,\gamma$)
\begin{enumerate}
\item \textbf{if }SAT($\{\gamma_{1}\cup\gamma_{2}\cup\ldots\gamma_{N}\}$)
\textbf{then return }$\{\gamma_{1}\cup\gamma_{2}\cup\ldots\gamma_{N}\}$
\item \textbf{while }$\phi=GetNextFormula()$ \textbf{do}
\begin{enumerate}
\item \textbf{For }\emph{i = }1 \textbf{up to} \emph{N}
\begin{enumerate}
\item $R=BeliefRevise(\beta_{i},\phi)$
\item \textbf{if $R\cap\gamma_{i}=\emptyset$ then goto 2}
\end{enumerate}
\item \textbf{end for}
\item \textbf{return $\phi$}
\end{enumerate}
\item \textbf{end while}
\item \textbf{return }NO\_SOLUTION
\end{enumerate}
\end{algorithm}

\noindent To find trivial solutions, we first check if all the agents
goals are satisfiable using a SAT solver. If all goals are satisfiable
together, then the solution is simply the concatenation of all the
goals into a single formula. If the goals are unsatisfiable, then
we enumerate the list of possible revision formulas. Each formula,
we revise the first agent's beliefs, and see if its goal is satisfied
by the result. If it is, repeat for the next agent. If the goal is
not satisfied, then the formula is not a valid solution, so exit that
iteration and get a new formula to test. If all agents are satisfied,
that formula is the announcement solution. If no possible formula
satisfies all agents, then the problem has no solution. \emph{GetNextFormula()
}returns the next formula in an \emph{$O(2^{M})$} brute force where
\emph{M} is the total number of variables in the set of agent beliefs
and goals. Belief Revision is performed through an external tool known
as GenC.

\section{Implementation}

\subsection{Input/Output}

Due to the complexity of the input data, relying on file I/O is difficult
from both an implementation and usage standpoint. To remedy this,
all I/O is performed through a GUI, implemented using Qt. All inputs
are human readable formula strings such as ``(1 and 2 or (3 and not
4))''. Variables are represented as numbers , and the logical operations
are limited to AND, OR, and NOT. This allows efficient representation
and input of fairly complex formulae from a user standpoint. Output
is displayed through pop-up message boxes displaying the relevant
information, be it an error, or the solution to the problem.

\subsection{Data Representation}

Internally, all formulas are represented as 2D arrays of integers,
where the integer is the variable number, and its sign represents
logical negation. All manipulation and I/O is based around this construct,
and is converted to and from other forms transparently by the application.
For enumerating the list of possible revision formulas, this is performed
as a brute force, where each bit is converted into a term for the
revision formula in DNF. For example, the binary string 1011 would
translate into ``1 and not 2 and 3 and 4''. This is how the \emph{GetNextFormula()}
function is implemented in section 3.

\section{Conclusion}

We present a tool for efficiently solving the problem of determining
shared belief announcement among multiple agents. We model the problem
as a deepest-node search inside a multitree, where the depth of a
node is the number of agents a given formula satisfies. Our hope is
to lay the groundwork for future algorithmic improvements based on
this initial problem model.
\end{document}
